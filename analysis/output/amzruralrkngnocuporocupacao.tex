\begin{table}[H]
\centering
\label{amzruralrkngnocuporocupacao}
\scalebox{0.70}{
\begin{threeparttable}
\caption{Variação absoluta do número de ocupações por tipo de ocupação entre 2012 e 2019}
\begin{tabular}{l*{2}{r}}
\midrule \midrule
                    &  Variação\\
\hline
Criadores de gado e trabalhadores qualificados da criação de gado&     113.926\\
Comerciantes de lojas&      37.405\\
Operadores de máquinas para elaborar alimentos e produtos afins&      22.138\\
Agricultores e trabalhadores qualificados no cultivo de hortas, viveiros e jardins&      17.932\\
Vendedores a domicilio&      17.761\\
Pescadores          &      13.767\\
Trabalhadores de limpeza de interior de edifícios, escritórios, hotéis e outros estabelecimentos&      12.937\\
Operadores de máquinas agrícolas e florestais móveis&      11.144\\
Vendedores de quiosques e postos de mercados&       8.724\\
Marinheiros de coberta e afins&       7.750\\
Cozinheiros         &       7.690\\
Trabalhadores dos serviços domésticos em geral&       5.741\\
Avicultores e trabalhadores qualificados da avicultura&       4.489\\
Vendedores ambulantes de serviços de alimentação&       4.054\\
Balconistas dos serviços de alimentação&       3.770\\
Ajudantes de professores&       3.661\\
Agricultores e trabalhadores qualificados de cultivos mistos&       3.616\\
Padeiros, confeiteiros e afins&       3.513\\
Cabeleireiros       &       3.462\\
Trabalhadores de cuidados pessoais a domicílios&       3.318\\
\bottomrule
\end{tabular}
\begin{tablenotes}
\item \scriptsize{Fonte: com base nos dados da PNAD Contínua, IBGE}
\end{tablenotes}
\end{threeparttable}
}
\end{table}
