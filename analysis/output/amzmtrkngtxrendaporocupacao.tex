\begin{table}[H]
\centering
\label{amzmtrkngtxrendaporocupacao}
\scalebox{0.60}{
\begin{threeparttable}
\caption{Taxas de crescimento de rendimentos médio por tipo de ocupação entre 2012 e 2019}
\begin{tabular}{l*{2}{r}}
\midrule \midrule
                    &Taxa de crescimento (\%)\\
\hline
Trabalhadores elementares da pesca e aquicultura&      807,92\\
Ferreiros e forjadores&      528,81\\
Trabalhadores da aquicultura&      401,77\\
Oficiais de bombeiro militar&      317,13\\
Apicultores, sericicultores e trabalhadores qualificados da apicultura e sericicultura&      251,89\\
Agricultores e trabalhadores qualificados de cultivos mistos&      230,22\\
Oficiais de polícia militar&      191,84\\
Dirigentes de empresas de construção&      187,65\\
Profissionais de nível médio de serviços estatísticos, matemáticos e afins&      183,16\\
Profissionais de relações públicas&      181,25\\
Músicos, cantores e compositores&      177,53\\
Ferramenteiros e afins&      160,46\\
Construtores de casas&      152,23\\
Técnicos em engenharia de minas e metalurgia&      149,42\\
Operadores de máquinas para fabricar cimento, pedras e outros produtos minerais&      145,62\\
Trabalhadores ambulantes dos serviços e afins&      145,19\\
Trabalhadores elementares da agropecuária&      144,47\\
Cartógrafos e agrimensores&      140,84\\
Trabalhadores encarregados de folha de pagamento&      138,28\\
Governantas e mordomos domésticos&      136,09\\
\bottomrule
\end{tabular}
\begin{tablenotes}
\item \scriptsize{Fonte: com base nos dados da PNAD Contínua, IBGE}
\end{tablenotes}
\end{threeparttable}
}
\end{table}
