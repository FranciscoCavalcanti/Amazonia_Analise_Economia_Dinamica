\begin{table}[H]
\centering
\label{amzmanausrkngtxrendaporocupacao}
\scalebox{0.70}{
\begin{threeparttable}
\caption{Taxas de crescimento de rendimentos médio por tipo de ocupação entre 2012 e 2019}
\begin{tabular}{l*{2}{r}}
\midrule \midrule
                    &Tx. Cresc. (\%)\\
\hline
Operadores de entrada de dados&      254,47\\
Técnicos e assistentes fisioterapeutas&      229,81\\
Trabalhadores do serviço de pessoal&      209,62\\
Desenhistas de produtos e  vestuário&      183,43\\
Gerentes de hotéis &      162,00\\
Guardiões de presídios&      157,29\\
Profissionais de nível médio de serviços estatísticos, matemáticos e afins&      154,05\\
Técnicos em ciências físicas e químicas&      152,89\\
Profissionais de vendas de tecnologia da informação e comunicações&      137,10\\
Oficiais de polícia militar&      123,33\\
Avicultores e trabalhadores qualificados da avicultura&      122,26\\
Operadores de teares e outras máquinas de tecelagem&      101,88\\
Despachantes aduaneiros&       96,14\\
Engenheiros mecânicos&       90,59\\
Dirigentes de serviços de saúde&       85,92\\
Inspetores de polícia e detetives&       84,25\\
Entrevistadores de pesquisas de mercado&       82,04\\
Trabalhadores dos serviços de informações&       77,24\\
Bombeiros           &       75,48\\
Técnicos agropecuários&       72,37\\
\bottomrule
\end{tabular}
\begin{tablenotes}
\item \scriptsize{Fonte: com base nos dados da PNAD Contínua, IBGE}
\end{tablenotes}
\end{threeparttable}
}
\end{table}
