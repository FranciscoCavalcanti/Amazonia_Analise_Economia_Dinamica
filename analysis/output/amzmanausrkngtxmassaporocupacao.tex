\begin{table}[H]
\centering
\label{amzmanausrkngtxmassaporocupacao}
\scalebox{0.70}{
\begin{threeparttable}
\caption{Taxas de crescimento de massa de rendimentos por tipo de ocupação entre 2012 e 2019}
\begin{tabular}{l*{2}{r}}
\midrule \midrule
                    &Tx. Cresc. (\%)\\
\hline
Vendedores por telefone&    1.230,68\\
Comerciantes de lojas&    1.100,85\\
Gerentes de hotéis &    1.068,68\\
Instaladores e reparadores em tecnologias da informação e comunicações&    1.057,78\\
Farmacêuticos      &      877,41\\
Profissionais de nível médio de serviços estatísticos, matemáticos e afins&      595,57\\
Outros professores de artes&      489,87\\
Dentistas auxiliares e ajudantes de odontologia&      402,96\\
Vendedores a domicilio&      401,95\\
Engenheiros mecânicos&      400,54\\
Engenheiros de meio ambiente&      381,63\\
Profissionais de vendas de tecnologia da informação e comunicações&      340,60\\
Secretários executivos e administrativos&      326,34\\
Oficiais de polícia militar&      323,74\\
Agricultores e trabalhadores qualificados no cultivo de hortas, viveiros e jardins&      317,69\\
Operadores de entrada de dados&      303,38\\
Trabalhadores de cuidados pessoais a domicílios&      291,21\\
Inspetores de polícia e detetives&      283,64\\
Trabalhadores do serviço de pessoal&      260,66\\
Profissionais de relações públicas&      259,92\\
\bottomrule
\end{tabular}
\begin{tablenotes}
\item \scriptsize{Fonte: com base nos dados da PNAD Contínua, IBGE}
\end{tablenotes}
\end{threeparttable}
}
\end{table}
