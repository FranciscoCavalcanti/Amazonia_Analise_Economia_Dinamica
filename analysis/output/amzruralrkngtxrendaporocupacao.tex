\begin{table}[H]
\centering
\label{amzruralrkngtxrendaporocupacao}
\scalebox{0.60}{
\begin{threeparttable}
\caption{Taxas de crescimento de rendimentos médio por tipo de ocupação entre 2012 e 2019}
\begin{tabular}{l*{2}{r}}
\midrule \midrule
                    &Taxa de crescimento (\%)\\
\hline
Joalheiros e lapidadores de gemas, artesãos de metais preciosos e semipreciosos&    2.594,71\\
Técnicos agropecuários&    1.077,47\\
Profissionais de vendas técnicas e médicas (exclusive tic)&      628,77\\
Técnicos em eletrônica&      531,65\\
Gerentes de hotéis &      487,61\\
Dirigentes de produção da aquicultura e pesca&      486,42\\
Cartógrafos e agrimensores&      399,70\\
Profissionais da proteção do meio ambiente&      389,04\\
Gerentes de sucursais de bancos, de serviços financeiros e de seguros&      275,08\\
Técnicos em engenharia de minas e metalurgia&      271,61\\
Supervisores de secretaria&      269,48\\
Supervisores de lojas&      260,97\\
Locutores de rádio, televisão e outros meios de comunicação&      233,40\\
Trabalhadores qualificados do tratamento de couros e peles&      205,11\\
Dirigentes de pesquisa e desenvolvimento&      187,41\\
Técnicos em engenharia mecânica&      184,60\\
Agentes imobiliários&      177,36\\
Dirigentes de publicidade e relações públicas&      176,01\\
Maquinistas de locomotivas&      168,67\\
Médicos gerais     &      160,56\\
\bottomrule
\end{tabular}
\begin{tablenotes}
\item \scriptsize{Fonte: com base nos dados da PNAD Contínua, IBGE}
\end{tablenotes}
\end{threeparttable}
}
\end{table}
