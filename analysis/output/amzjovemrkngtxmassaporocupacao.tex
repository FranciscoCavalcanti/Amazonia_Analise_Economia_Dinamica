\begin{table}[H]
\centering
\label{amzjovemrkngtxmassaporocupacao}
\scalebox{0.70}{
\begin{threeparttable}
\caption{Taxas de crescimento de massa de rendimentos por tipo de ocupação entre 2012 e 2019}
\begin{tabular}{l*{2}{r}}
\midrule \midrule
                    &Tx. Cresc. (\%)\\
\hline
Ajudantes de professores&      852,00\\
Trabalhadores e assistentes sociais de nível médio&      646,38\\
Oficiais maquinistas em navegação&      525,96\\
Atletas e esportistas&      459,17\\
Arquitetos de edificações&      428,72\\
Instaladores e reparadores em tecnologias da informação e comunicações&      413,34\\
Trabalhadores de cuidados pessoais a domicílios&      399,50\\
Vendedores por telefone&      382,68\\
Oficiais das forças armadas&      363,24\\
Instaladores de material isolante térmico e acústico&      340,89\\
Balconistas dos serviços de alimentação&      316,93\\
Vendedores ambulantes de serviços de alimentação&      272,16\\
Construtores de casas&      258,66\\
Outros professores de idiomas&      245,08\\
Costureiros, bordadeiros e afins&      223,67\\
Cartógrafos e agrimensores&      216,29\\
Mecânicos-instaladores de sistemas de refrigeração e climatização&      213,79\\
Dentistas auxiliares e ajudantes de odontologia&      211,72\\
Dirigentes financeiros&      194,40\\
Cuidadores de animais&      190,20\\
\bottomrule
\end{tabular}
\begin{tablenotes}
\item \scriptsize{Fonte: com base nos dados da PNAD Contínua, IBGE}
\end{tablenotes}
\end{threeparttable}
}
\end{table}
