\begin{table}[H]
\centering
\label{rkngtxocuporocupacao}
\scalebox{0.60}{
\begin{threeparttable}
\caption{Taxas de crescimento de ocupações por tipo de ocupação entre 2012 e 2019}
\begin{tabular}{l*{2}{r}}
\midrule \midrule
                    &Taxa de crescimento (\%)\\
\hline
Outras ocupações elementares não classificadas anteriormente&   11.908,76\\
Alfaiates, modistas, chapeleiros e peleteiros&    6.458,93\\
Vendedores não classificados anteriormente&      932,67\\
Vendedores por telefone&      919,38\\
Artesãos não classificados anteriormente&      855,29\\
Profissionais em direito não classificados anteriormente&      845,18\\
Dentistas auxiliares e ajudantes de odontologia&      765,26\\
Agricultores e trabalhadores qualificados de cultivos mistos&      734,44\\
Ajudantes de professores&      657,48\\
Trabalhadores e assistentes sociais de nível médio&      633,71\\
Operadores de máquinas e de instalações fixas não classificados anteriormente&      533,55\\
Outros trabalhadores qualificados e operários da construção não classificados anteriormente&      425,36\\
Cuidadores de animais&      404,41\\
Trabalhadores qualificados da preparação do fumo e seus produtos&      388,12\\
Instaladores e reparadores em tecnologias da informação e comunicações&      352,30\\
Trabalhadores de cuidados pessoais a domicílios&      337,70\\
Dirigentes financeiros&      321,14\\
Construtores de casas&      289,40\\
Vendedores a domicilio&      289,00\\
Arquitetos de edificações&      274,67\\
\bottomrule
\end{tabular}
\begin{tablenotes}
\item \scriptsize{Fonte: com base nos dados da PNAD Contínua, IBGE}
\end{tablenotes}
\end{threeparttable}
}
\end{table}
