\begin{table}[H]
\centering
\label{amzmanausrkngnocuporocupacao}
\scalebox{0.70}{
\begin{threeparttable}
\caption{Variação absoluta do número de ocupações por tipo de ocupação entre 2012 e 2019}
\begin{tabular}{l*{2}{r}}
\midrule \midrule
                    &  Variação\\
\hline
Comerciantes de lojas&      87.005\\
Vendedores a domicilio&      62.584\\
Escriturários gerais&      31.635\\
Vendedores ambulantes de serviços de alimentação&      24.460\\
Vendedores de quiosques e postos de mercados&      23.345\\
Trabalhadores florestais elementares&      17.833\\
Cozinheiros         &      16.381\\
Trabalhadores de limpeza de interior de edifícios, escritórios, hotéis e outros estabelecimentos&      14.566\\
Padeiros, confeiteiros e afins&      13.206\\
Profissionais de nível médio de enfermagem&      12.993\\
Balconistas dos serviços de alimentação&      12.012\\
Criadores de gado e trabalhadores qualificados da criação de gado&      10.988\\
Caixas e expedidores de bilhetes&      10.289\\
Mecânicos e reparadores de veículos a motor&      10.138\\
Condutores de automóveis, taxis e caminhonetes&       9.840\\
Trabalhadores de cuidados pessoais a domicílios&       9.202\\
Porteiros e zeladores&       8.912\\
Condutores de motocicletas&       8.269\\
Pintores e empapeladores&       7.782\\
Advogados e juristas&       7.753\\
\bottomrule
\end{tabular}
\begin{tablenotes}
\item \scriptsize{Fonte: com base nos dados da PNAD Contínua, IBGE}
\end{tablenotes}
\end{threeparttable}
}
\end{table}
