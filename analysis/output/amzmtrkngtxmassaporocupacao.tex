\begin{table}[H]
\centering
\label{amzmtrkngtxmassaporocupacao}
\scalebox{0.70}{
\begin{threeparttable}
\caption{Taxas de crescimento de massa de rendimentos por tipo de ocupação entre 2012 e 2019}
\begin{tabular}{l*{2}{r}}
\midrule \midrule
                    &Tx. Cresc. (\%)\\
\hline
Alfaiates, modistas, chapeleiros e peleteiros&    1.359,27\\
Trabalhadores e assistentes sociais de nível médio&    1.126,47\\
Construtores de casas&      976,68\\
Dirigentes de explorações de mineração&      961,05\\
Instaladores de material isolante térmico e acústico&      935,09\\
Ajudantes de professores&      649,11\\
Dirigentes de políticas e planejamento&      601,19\\
Desenvolvedores de programas e aplicativos (software)&      593,30\\
Instrutores de educação física e atividades recreativas&      580,36\\
Analistas financeiros&      560,40\\
Agricultores e trabalhadores qualificados de cultivos mistos&      550,23\\
Graduados e praças da polícia militar&      548,29\\
Dentistas auxiliares e ajudantes de odontologia&      540,02\\
Trabalhadores de cuidados pessoais a domicílios&      515,09\\
Profissionais de relações públicas&      512,80\\
Vendedores a domicilio&      449,27\\
Escritores          &      428,31\\
Cuidadores de animais&      421,69\\
Músicos, cantores e compositores&      387,20\\
Oficiais de bombeiro militar&      374,83\\
\bottomrule
\end{tabular}
\begin{tablenotes}
\item \scriptsize{Fonte: com base nos dados da PNAD Contínua, IBGE}
\end{tablenotes}
\end{threeparttable}
}
\end{table}
