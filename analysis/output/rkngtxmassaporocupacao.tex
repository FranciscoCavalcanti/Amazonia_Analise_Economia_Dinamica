\begin{table}[H]
\centering
\label{rkngtxmassaporocupacao}
\scalebox{0.70}{
\begin{threeparttable}
\caption{Taxas de crescimento da massa de rendimentos por tipo de ocupação entre 2012 e 2019}
\begin{tabular}{l*{2}{r}}
\midrule \midrule
                    &Taxa de crescimento (\%)\\
\hline
Alfaiates, modistas, chapeleiros e peleteiros&    3.902,42\\
Ajudantes de professores&      692,04\\
Vendedores por telefone&      659,67\\
Dentistas auxiliares e ajudantes de odontologia&      618,00\\
Dirigentes de serviços de saúde&      608,64\\
Trabalhadores e assistentes sociais de nível médio&      542,28\\
Oficiais maquinistas em navegação&      535,64\\
Construtores de casas&      431,62\\
Cuidadores de animais&      341,28\\
Entrevistadores de pesquisas de mercado&      338,97\\
Instaladores de material isolante térmico e acústico&      326,41\\
Dirigentes financeiros&      298,08\\
Profissionais da proteção do meio ambiente&      296,51\\
Trabalhadores de cuidados pessoais a domicílios&      293,65\\
Instaladores e reparadores em tecnologias da informação e comunicações&      262,64\\
Vendedores a domicilio&      260,93\\
Balconistas dos serviços de alimentação&      260,84\\
Atletas e esportistas&      255,36\\
Gerentes de centros esportivos,  de diversão e culturais&      224,39\\
Instrutores de educação física e atividades recreativas&      222,16\\
\bottomrule
\end{tabular}
\begin{tablenotes}
\item \scriptsize{Fonte: com base nos dados da PNAD Contínua, IBGE}
\end{tablenotes}
\end{threeparttable}
}
\end{table}
