\begin{table}[H]
\centering
\label{amzurbanarkngnocuporocupacao}
\scalebox{0.60}{
\begin{threeparttable}
\caption{Variação absoluta do número de ocupações por tipo de ocupação entre 2012 e 2019}
\begin{tabular}{l*{2}{r}}
\midrule \midrule
                    &  Variação\\
\hline
Comerciantes de lojas&     257.018\\
Vendedores a domicilio&     175.321\\
Escriturários gerais&      93.657\\
Alfaiates, modistas, chapeleiros e peleteiros&      65.128\\
Vendedores não classificados anteriormente&      62.620\\
Vendedores ambulantes de serviços de alimentação&      55.905\\
Vendedores de quiosques e postos de mercados&      49.147\\
Mecânicos e reparadores de veículos a motor&      45.187\\
Cozinheiros         &      45.066\\
Trabalhadores de limpeza de interior de edifícios, escritórios, hotéis e outros estabelecimentos&      38.071\\
Profissionais de nível médio de enfermagem&      36.317\\
Condutores de automóveis, taxis e caminhonetes&      34.061\\
Especialistas em tratamento de beleza e afins&      32.688\\
Outras ocupações elementares não classificadas anteriormente&      32.470\\
Balconistas dos serviços de alimentação&      27.596\\
Padeiros, confeiteiros e afins&      27.283\\
Professores do ensino pré-escolar&      26.256\\
Trabalhadores elementares da indústria de transformação não classificados anteriormente&      25.181\\
Trabalhadores de cuidados pessoais a domicílios&      24.452\\
Repositores de prateleiras&      23.869\\
\bottomrule
\end{tabular}
\begin{tablenotes}
\item \scriptsize{Fonte: com base nos dados da PNAD Contínua, IBGE}
\end{tablenotes}
\end{threeparttable}
}
\end{table}
