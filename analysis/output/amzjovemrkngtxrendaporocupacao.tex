\begin{table}[H]
\centering
\label{amzjovemrkngtxrendaporocupacao}
\scalebox{0.60}{
\begin{threeparttable}
\caption{Taxas de crescimento de rendimentos médio por tipo de ocupação entre 2012 e 2019}
\begin{tabular}{l*{2}{r}}
\midrule \midrule
                    &Taxa de crescimento (\%)\\
\hline
Oficiais maquinistas em navegação&      890,00\\
Agentes da administração pública para aplicação da lei e afins não classificados anteriormente&      265,66\\
Cartógrafos e agrimensores&      242,44\\
Gerentes de serviços não classificados anteriormente&      213,36\\
Profissionais em direito não classificados anteriormente&      209,26\\
Técnicos agropecuários&      204,06\\
Trabalhadores de funerárias e embalsamadores&      152,28\\
Capitães, oficiais de coberta e práticos&      144,74\\
Tradutores, intérpretes e linguistas&      124,72\\
Limpadores de fachadas&      122,41\\
Costureiros, bordadeiros e afins&      119,13\\
Inspetores de polícia e detetives&      118,41\\
Chefes de cozinha   &       99,70\\
Secretários executivos e administrativos&       86,37\\
Operadores de teares e outras máquinas de tecelagem&       84,37\\
Despachantes aduaneiros&       77,06\\
Outros professores de idiomas&       74,17\\
Educadores para necessidades especiais&       70,80\\
Instaladores e reparadores de linhas elétricas&       69,76\\
Outros professores de artes&       69,63\\
\bottomrule
\end{tabular}
\begin{tablenotes}
\item \scriptsize{Fonte: com base nos dados da PNAD Contínua, IBGE}
\end{tablenotes}
\end{threeparttable}
}
\end{table}
