\begin{table}[H]
\centering
\label{amzurbanarkngtxocuporocupacao}
\scalebox{0.70}{
\begin{threeparttable}
\caption{Taxas de crescimento de ocupações por tipo de ocupação entre 2012 e 2019}
\begin{tabular}{l*{2}{r}}
\midrule \midrule
                    &Tx. Cresc. (\%)\\
\hline
Alfaiates, modistas, chapeleiros e peleteiros&    6.089,57\\
Vendedores por telefone&      884,25\\
Dentistas auxiliares e ajudantes de odontologia&      758,75\\
Trabalhadores e assistentes sociais de nível médio&      642,49\\
Ajudantes de professores&      570,83\\
Cuidadores de animais&      359,84\\
Instaladores e reparadores em tecnologias da informação e comunicações&      353,61\\
Dirigentes financeiros&      346,84\\
Trabalhadores de cuidados pessoais a domicílios&      309,39\\
Construtores de casas&      293,53\\
Vendedores a domicilio&      291,19\\
Arquitetos de edificações&      275,67\\
Vendedores ambulantes de serviços de alimentação&      239,85\\
Técnicos e assistentes fisioterapeutas&      230,95\\
Balconistas dos serviços de alimentação&      230,36\\
Atletas e esportistas&      183,73\\
Instaladores de material isolante térmico e acústico&      173,05\\
Engenheiros industriais e de produção&      172,42\\
Dinamitadores e detonadores&      170,46\\
Comerciantes de lojas&      169,24\\
\bottomrule
\end{tabular}
\begin{tablenotes}
\item \scriptsize{Fonte: com base nos dados da PNAD Contínua, IBGE}
\end{tablenotes}
\end{threeparttable}
}
\end{table}
