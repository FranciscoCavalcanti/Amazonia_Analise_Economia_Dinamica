\begin{table}[H]
\centering
\label{amzruralrkngtxmassaporocupacao}
\scalebox{0.70}{
\begin{threeparttable}
\caption{Taxas de crescimento de massa de rendimentos por tipo de ocupação entre 2012 e 2019}
\begin{tabular}{l*{2}{r}}
\midrule \midrule
                    &Tx. Cresc. (\%)\\
\hline
Técnicos em engenharia de minas e metalurgia&    2.774,81\\
Agricultores e trabalhadores qualificados de cultivos mistos&    1.777,52\\
Ajudantes de professores&    1.595,51\\
Trabalhadores de cuidados pessoais a domicílios&    1.500,71\\
Dirigentes de pesquisa e desenvolvimento&    1.415,13\\
Outros professores de idiomas&    1.335,75\\
Graduados e praças da polícia militar&      999,23\\
Maquinistas de locomotivas&      764,82\\
Profissionais da proteção do meio ambiente&      655,63\\
Dirigentes de publicidade e relações públicas&      554,19\\
Técnicos em engenharia mecânica&      547,51\\
Marinheiros de coberta e afins&      535,42\\
Professores de universidades e do ensino superior&      506,52\\
Gerentes de hotéis &      457,21\\
Médicos gerais     &      431,43\\
Vendedores a domicilio&      403,12\\
Graduados e praças do corpo de bombeiros&      387,20\\
Trabalhadores da aquicultura&      367,49\\
Trabalhadores qualificados do tratamento de couros e peles&      356,60\\
Operadores de máquinas de lavar, tingir e passar roupas&      266,81\\
\bottomrule
\end{tabular}
\begin{tablenotes}
\item \scriptsize{Fonte: com base nos dados da PNAD Contínua, IBGE}
\end{tablenotes}
\end{threeparttable}
}
\end{table}
