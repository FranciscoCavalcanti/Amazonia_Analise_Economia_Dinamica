\begin{table}[H]
\centering
\label{amzjovemrkngtxocuporocupacao}
\scalebox{0.70}{
\begin{threeparttable}
\caption{Taxas de crescimento de ocupações por tipo de ocupação entre 2012 e 2019}
\begin{tabular}{l*{2}{r}}
\midrule \midrule
                    &Tx. Cresc. (\%)\\
\hline
Alfaiates, modistas, chapeleiros e peleteiros&    1.034,70\\
Ajudantes de professores&      875,63\\
Arquitetos de edificações&      656,25\\
Trabalhadores e assistentes sociais de nível médio&      605,57\\
Vendedores por telefone&      556,21\\
Instaladores e reparadores em tecnologias da informação e comunicações&      506,01\\
Trabalhadores de cuidados pessoais a domicílios&      392,34\\
Oficiais das forças armadas&      366,77\\
Atletas e esportistas&      345,63\\
Balconistas dos serviços de alimentação&      311,25\\
Vendedores ambulantes de serviços de alimentação&      308,81\\
Dentistas           &      278,17\\
Dentistas auxiliares e ajudantes de odontologia&      262,82\\
Vendedores a domicilio&      260,94\\
Dirigentes financeiros&      231,91\\
Instaladores de material isolante térmico e acústico&      230,45\\
Comerciantes de lojas&      215,60\\
Construtores de casas&      194,69\\
Engenheiros químicos&      181,55\\
Outros trabalhadores de limpeza&      175,39\\
\bottomrule
\end{tabular}
\begin{tablenotes}
\item \scriptsize{Fonte: com base nos dados da PNAD Contínua, IBGE}
\end{tablenotes}
\end{threeparttable}
}
\end{table}
