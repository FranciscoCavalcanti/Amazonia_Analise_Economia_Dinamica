\begin{table}[H]
\centering
\label{amzmtrkngtxrendaporatividade}
\scalebox{0.60}{
\begin{threeparttable}
\caption{Taxas de crescimento de rendimentos médio por grupamentos de atividade entre 2012 e 2019}
\begin{tabular}{l*{2}{r}}
\midrule \midrule
                    &Taxa de crescimento (\%)\\
\hline
Fabricação de produtos derivados do petróleo&      423,75\\
Apicultura          &      382,17\\
Produção de sementes e mudas certificadas&      257,28\\
Aqüicultura        &      237,07\\
Fabricação de cabines, carrocerias, reboques e peças para veículos automotores&      198,73\\
Atividades de organizações associativas patronais, empresariais e profissionais&      163,95\\
Produção e distribuição de combustíveis gasosos por redes urbanas&      132,56\\
Fabricação e montagem de veículos automotores&      122,35\\
Outros serviços coletivos prestados pela administração pública - Municipal&      119,92\\
Extração de carvão mineral&      119,30\\
Metalurgia dos metais não-ferrosos&      112,36\\
Atividades de condicionamento físico&      111,07\\
Fabricação de embalagens e de produtos diversos de papel, cartolina, papel-cartão e papelão ondulado&      107,05\\
Lavanderias, tinturarias e toalheiros&       98,87\\
Fabricação de tintas, vernizes, esmaltes, lacas e produtos afins&       96,51\\
Extração de gemas (pedras preciosas e semi-preciosas)&       91,42\\
Atividades de televisão&       83,15\\
Agropecuária       &       81,42\\
Extração de minérios de metais preciosos&       65,26\\
Extração de petróleo e gás natural&       63,90\\
\bottomrule
\end{tabular}
\begin{tablenotes}
\item \scriptsize{Fonte: com base nos dados da PNAD Contínua, IBGE}
\end{tablenotes}
\end{threeparttable}
}
\end{table}
